\section{Введение} 

\par За последние несколько лет задача восстановления матрицы по части ее элементов привлекала большое внимание. Работы
     \cite{6}, \cite{7}, \cite{13}, \cite{21}, посвящены решению этой проблемы путем минимизации выпуклого функционала, 
     представляющим собой ядерную норму матрицы. Одним из важных результатов этих работ является оценка на число элементов,
     необходимое для полного восстановления матрицы размера $n \times n$ - $O(nr\max\{\mu_0, \mu_1^2\}\log^2(n))$, где 
     $r$ - ранг матрицы, а  $\mu_0, \ \mu_1$ - параметры когерентности. В свою очередь, Кандес и Тао доказали, что 
     $O(nr\mu_0\log(n))$ известных элементов являются необходимым условием для того, чтобы минимизация ядерной нормы была 
     близка к оптимальному решению исходной задачи. Представленный метод для восстановления матрицы по ее элементам
     теоретически обоснован и широко используется в приложениях.
\par В многомерном случае данный вопрос изучен менее подробно. Одной из причин этому является то, что задача нахождения ядерной 
     нормы для тензора принадлежит NP классу (\cite{32}), и  поэтому минимизация соответствующего функционала является очень
     трудоемкой в плане компьютерных вычислений. В двумерном случае исходная проблема сводится к минимизации ядерной нормы
     матрицы, и как правило, тензорное обобщение данной нормы часто представляет собой линейные комбинации норм его
     разверток.	Существующие нижние оценки на число известных элементов, необходимое для надежного восстановления 
     тензора показывают неоптимальность выбранных моделей.
\par В данной работе рассмотрены теоретические сведения о методах восстановления матриц и тензоров по части их элементов, 
    а также приведены соответствующие численные эксперименты.